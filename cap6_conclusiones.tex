% -*-cap1.tex-*-
% Este fichero es parte de la plantilla LaTeX para
% la realización de Proyectos Final de Carrera, protejido
% bajo los términos de la licencia GFDL.
% Para más información, la licencia completa viene incluida en el
% fichero fdl-1.3.tex

% Copyright (C) 2009 Pablo Recio Quijano 

%\section{Introducción}

En el presente documento se ha realizado una revisión formal de la literatura para responder a las preguntas de la investigación relacionadas con la evaluación automatizada de competencias genéricas mediante el uso de entornos de aprendizaje virtuales.

Los criterios de inclusión de artículos apenas nos dejaron un 7\% de los que fueron preseleccionados tras el proceso de búsqueda llevado a cabo en las diferentes bibliotecas digitales. La selección primaria ha sido clasificada en base al tipo de contribución que realizan a la ciencia, en base al tipo de investigación y en base al ámbito de aplicación del trabajo. Esta clasificación ha sido representada mediante diferentes tablas y figuras.

El escaso número de resultados obtenido viene a corroborar que es un tema de reciente interés, siendo además destacable que el grueso de artículos rescatado data de los últimos 3 años. No obstante, no se ha encontrado un trabajo de las características del que tenemos intención de realizar. Las competencias genéricas se evalúan mediante el uso de herramientas de tipo LMS, Wiki o Redes Sociales, pero casi siempre es necesario el factor humano para evaluar o estimar los indicadores de competencias. Normalmente mediante el uso de una rúbrica o mediante la revisión de una consecución satisfactoria o no de objetivos del curso.

Las preguntas de investigación planteadas fueron respondidas encontrando trabajos asociados a cada una de ellas. Aunque como se indica en el párrafo anterior, los trabajos encontrados evalúan competencias genéricas, aunque no lo hacen desde una perspectiva tan automatizada como se pretende que sea nuestro trabajo.

Se presentan las líneas de trabajo futuro, basadas en el desarrollo de la herramienta EvalCourse para la evaluación automática de indicadores de los LMS. En la actualidad hay un proyecto de Innovación Docente coordinado por el autor de este trabajo y que junto con este trabajo forman las bases de la que será su tesis doctoral.




