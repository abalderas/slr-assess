% -*-cap1.tex-*-
% Este fichero es parte de la plantilla LaTeX para
% la realización de Proyectos Final de Carrera, protejido
% bajo los términos de la licencia GFDL.
% Para más información, la licencia completa viene incluida en el
% fichero fdl-1.3.tex

% Copyright (C) 2009 Pablo Recio Quijano 

%\section{Introducción}

En el presente documento se ha realizado una revisión formal de la literatura para responder a las preguntas de la investigación relacionadas con la evaluación automatizada de competencias genéricas mediante el uso de entornos de aprendizaje virtuales. Esta revisión formal se ha desarrollado siguiendo las directrices publicadas en la metodología propuesta por Kitchenham.

Se identificaron los términos que se utilizarían en las principales bibliotecas digitales y se realizó la búsqueda para hallar trabajos que ayudasen a responder a las preguntas de investigación que se plantearon. Los criterios de inclusión de artículos nos dejaron un 7\% de los que fueron preseleccionados tras este proceso de búsqueda. La selección primaria ha sido clasificada en base al tipo de contribución que realizan a la ciencia, en base al tipo de investigación y en base al ámbito de aplicación del trabajo. Esta clasificación ha sido representada mediante diferentes tablas y figuras.

El escaso número de resultados obtenido viene a corroborar la escasez de colaboraciones entre ciencias de la educación e informática. El grueso de artículos rescatado data de los últimos 3 años. De los trabajos encontrados, la tendencia dominante con un 43,8\% es la de delegar en el alumno, ya sea mediante autoevaluación, o mediante evaluación entre iguales, la corrección de las actividades que evidencian el desempeño de una competencia. Además, la segunda tendencia, con un 21,9\% es la que se basa en rúbricas para refrendarlo. Al final, este método también delega en una persona o conjunto de personas, sólo que esta vez es el profesor el encargado de corregirlo. Es decir, un 65,7\% de los trabajos encontrados necesita de la intervención humana para calcular las puntuaciones, por lo que el procedimiento no es automático, lo que de algún u otro modo, podría incurrir en tareas no escalables para el docente.

Por otro lado, están las herramientas GBL, con un 18,8\%, que aunque automatizan la evaluación del desempeño en el juego de un alumno, no están integradas con el sistema de evaluación, por lo que traspasar las notas a la plataforma de la asignatura es una tarea que de nuevo implica al docente. Finalmente, queda un 15,6\% de trabajos de diversa índole, en los que se combinan herramientas para la evaluación de competencias y se realizan revisiones de otros trabajos.

Estos trabajos se utilizaron para responder a cada una de las preguntas de investigación. Como se indicaba en el párrafo anterior, los trabajos encontrados evalúan competencias genéricas, aunque no lo hacen desde una perspectiva tan automatizada como se pretende que sea nuestro trabajo.

Se presentan las líneas de trabajo futuro, basadas en el desarrollo de la herramienta EvalCourse para la evaluación automática de indicadores de los LMS. En la actualidad hay un proyecto de Innovación Docente coordinado por el autor de este trabajo y que junto con este trabajo forman las bases de la que será su tesis doctoral.




