% -*-cap1.tex-*-
% Este fichero es parte de la plantilla LaTeX para
% la realización de Proyectos Final de Carrera, protejido
% bajo los términos de la licencia GFDL.
% Para más información, la licencia completa viene incluida en el
% fichero fdl-1.3.tex

% Copyright (C) 2009 Pablo Recio Quijano 

%\section{Introducción}

El interés en la evaluación del aprendizaje ha cambiado del conocimiento a las competencias. Hoy en día, las universidades y las empresas van cada vez más de la mano en lo referente a la formación del alumnado. Las empresas buscan profesionales capaces de renovar constantemente sus conocimientos y competencias. Por tanto, ser capaces de desarrollar y evaluar competencias se convierte en fundamental dentro del nuevo paradigma educativo. Las competencias de aprendizaje son habilidades que un alumno ha de ser capaz de demostrar una vez que termina su formación. Estas competencias de aprendizaje se dividen en dos grupos: específicas y genéricas.

El ámbito de nuestro trabajo son las plataformas virtuales de tipo LMS ó VLE empleadas hoy día en los centros educativos a todos los niveles. Cuando el número de alumnos es elevado, como ocurre en los MOOC, se vuelve muy poco escalable la tarea de realizar la evaluación por parte del docente. Si en ocasiones es difícil para el profesor conseguir cumplir la planificación fijada para un curso, más complicado será si ha de añadir actividades para evaluar competencias genéricas. Por tanto, sino se automatiza la evaluación, se presentarán evidentes problemas de escalabilidad.

Se ha realizado un estudio con el fin de localizar trabajos relacionados con la tecnología para la mejora del aprendizaje (TEL) para la evaluación de competencias de manera automática. Los LMS ó VLE almacenan información de estudiantes, profesores, cursos, tareas, trabajos, etc. Estos elementos se relacionan y configuran para ofrecer al usuario una experiencia de curso virtual. Haciendo todos estos datos almacenados accesibles, se podría sacar partido a los mismos con fines científicos.

En el presente documento se ha realizado una revisión formal de la literatura siguiendo las directrices publicadas en la metodología propuesta por Kitchenham. La revisión parte de una serie de preguntas de investigación cuyo planteamiento tuvo como fin saber qué competencias se han evaluado, con qué herramientas y qué técnicas de forma automática o asistida por ordenador, y si se han utilizado registros de actividad para ello.

Se identificaron los términos que se han utilizado en las principales bibliotecas digitales y se realizó la búsqueda para hallar trabajos que ayudasen a responder a las preguntas de investigación. Los criterios de inclusión de artículos nos dejaron un 7\% de los que fueron preseleccionados tras este proceso de búsqueda. La selección primaria ha sido clasificada en base al tipo de contribución que realizan a la ciencia, en base al tipo de investigación y en base al ámbito de aplicación del trabajo. Esta clasificación ha sido representada mediante diferentes tablas y figuras.

El escaso número de resultados obtenido viene a corroborar la escasez de colaboraciones entre ciencias de la educación e informática. El grueso de artículos rescatado data de los últimos 3 años. De los trabajos encontrados, la tendencia dominante con un 43,8\% es la de delegar en el alumno, ya sea mediante autoevaluación, o mediante evaluación entre iguales, la corrección de las actividades que evidencian el desempeño de una competencia. Además, la segunda tendencia, con un 21,9\% es la que se basa en rúbricas para refrendarlo. Al final, este método también delega en una persona o conjunto de personas, sólo que esta vez es el profesor el encargado de corregirlo. Es decir, un 65,7\% de los trabajos encontrados necesita de la intervención humana para calcular las puntuaciones, por lo que el procedimiento no es automático, lo que de algún u otro modo, podría incurrir en tareas no escalables para el docente.

Por otro lado, están las herramientas GBL, con un 18,8\%, que aunque automatizan la evaluación del desempeño en el juego de un alumno, no están integradas con el sistema de evaluación, por lo que traspasar las notas a la plataforma de la asignatura es una tarea que de nuevo implica al docente. Finalmente, queda un 15,6\% de trabajos de diversa índole, en los que se combinan herramientas para la evaluación de competencias y se realizan revisiones de otros trabajos.

Aunque hemos encontrado en los trabajos respuesta a todas las preguntas, no hay una tendencia predominante en ninguna de las áreas, y además, ninguna de las soluciones planteadas resuelve la problemática principal de nuestro trabajo de investigación, es decir, solucionar el problema de la escalabilidad mediante la automatización, ya que siempre se delega de alguna u otra forma en el usuario (docente o alumno), o las herramientas no están integradas. Esto muestra una necesidad de profundizar en la automatización en la evaluación de competencias, lo que sienta las bases de la línea en que apunta mi tesis doctoral. Para ello por un lado, se plantea como objetivo funcional, que hemos de ser capaces de recoger información de una diversidad de sistemas: LMS, wikis y CMS de manera automática; y por otro lado, como objetivo no funcional, hemos de ser capaces de evaluar el desempeño de los alumnos en el sistema, a pesar de que el número de alumnos sea muy elevado.




