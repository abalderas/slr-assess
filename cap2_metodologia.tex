% -*-cap2_metodologia.tex-*-

Un \emph{Estudio Sistemático de Mapeo} (SMS, Systematic Mapping Study), es una amplia revisión de los estudios primarios en un area específica cuyo objetivo es identificar alguna evidencia sobre el tema. Este estudio se basa en las directrices publicadas en la metodología propuesta por Kitchenham \cite{Kitchenham:2010}. Esta metodología describe cómo se deben planificar, ejecutar y presentar los resultados de una revisión de la literatura en ingeniería del software. Para este trabajo se ha utilizado la propuesta de Petersen \cite{Petersen:2008}.

Es importante seguir un procedimiento sistemático para llevar a cabo una revisión rigurosa de la literatura. En un primer momento, se trató de seguir la clasificación aportada por \cite{Redecker:2012} para el ámbito de aplicación de la contribución. En él se plantea una división generacional:
\begin{itemize}
\item Primera generación: administración y calificación automatizada de pruebas convencionales
\item Segunda generación: pruebas adaptativas
\item Tercera generación: evaluación integrada continua
\item Cuarta generación: retroalimentación tutorizada y personalizada
\end{itemize}
Sin embargo, la selección de la literatura realizada en este trabajo, no se adaptaba a esta categorización, por lo que finalmente los artículos fueron agrupados según el tipo de herramienta, modelo, paradigma o discusión que planteaban. La clasificación final fue:
\begin{itemize}
\item Course’s Learning Outcomes (CLO) and rubrics: trabajos que evalúan las competencias genéricas de los alumnos a partir de rúbricas de los resultados del trabajo de los alumnos
\item Peer and self eAssessment: trabajos que delegan la función de evaluar competencias genéricas en los compañeros o en la autoevaluación
\item Game-Based Learning (GBL): trabajos que se basan en juegos para la evaluación de competencias genéricas
\item eAssessment and reviews: resto de trabajos que agrupan propuestas para la evaluación de competencias genéricas mediante el uso de la tecnología 
\end{itemize}

\section{Justificación}
El motivo de este trabajo, tal como se describió en la introducción, es la localización de trabajos relacionados con la tecnología para la mejora del aprendizaje (TEL) cuyo fin sea la evaluación de competencias de manera automática. En la era de la Web 2.0, se está popularizando la integración y personalización de contenidos web de diferentes fuentes. Esta integración y personalización podría impulsar la implementación de pasarelas con fines de educación avanzada, y así convertir los datos en accesibles y reutilizables \cite{Wenjun:2008}.

Los LMS ó VLE almacenan información de estudiantes, profesores, cursos, tareas, trabajos, etc. Estos elementos se relacionan y configuran para ofrecer al usuario una experiencia de curso virtual. Haciendo todos estos datos almacenados accesibles, se podría sacar partido a los mismos con fines científicos. Y tal como se comentó en la introducción, para que esta evaluación de competencias sea escalable y más aún cuando el número de alumnos es importante, no queda otra alternativa que hacerlo de manera automatizada.

% Añadir despues de con fines científicos \cite{PENDIENTE:CITA:REGISTROS:ALMACENADOS:LMS}

\section{Preguntas de investigación}

En este punto, hemos de plantearnos cuáles son las cuestiones que necesitamos resolver para llevar a cabo nuestra investigación. Necesitamos saber en qué punto se haya el desarrollo por parte de la comunidad científica en lo que se refiere al uso de los registros de los entornos de aprendizaje para evaluar las competencias genéricas. Para ello planteamos la primera pregunta: \emph{¿Cuáles son las competencias que se han evaluado de forma automática o asistida por ordenador a partir del uso de los entornos virtuales?}

Además, necesitamos saber de qué forma lo hace, qué métodos y técnicas se utilizan para este fin. Para eso se plantean las preguntas segunda (\emph{¿Qué herramientas o metodologías se utilizan para evaluar competencias mediante el uso de los entornos virtuales?}) y tercera (\emph{¿Con qué técnicas se pueden obtener evidencias objetivas del desarrollo de las siguientes competencias en un entorno virtual?}).

Finalmente se necesita saber si los registros de interacción de un sistema LMS se están utilizando para la evaluación: \emph{¿Son utilizados los registros de actividad de los entornos virtuales para la evaluación?}.

%Finalmente se necesita saber si los resultados que se obtengan de los registros de interacción de un sistema LMS realmente reflejan el desempeño o no de los estudiantes en qué competencia, planteádose por tanto la cuarta pregunta: \emph{¿Los resultados que se obtendrían de la evaluación de competencias genéricas de los alumnos mediante el uso de los registros de actividad de los entornos virtuales son indicadores fiables del desempeño de dichas competencias?}.

\bigskip
En resumen, las preguntas de investigación son las siguientes:
\begin{enumerate}
%\item ¿Cuáles son las competencias que se han evaluado de forma automática o asistida por ordenador a partir del uso de los LMS?
%\item ¿Qué métodos y técnicas se utilizan para evaluar competencias mediante el uso de un LMS?
%\item ¿Con qué técnicas se pueden obtener evidencias (numéricas) objetivas del desarrollo de las siguientes competencias en un LMS?
%\item ¿Los resultados que se obtendrían de la evaluación de competencias genéricas de los alumnos mediante el uso de los registros de actividad de los LMS son indicadores fiables del desempeño de dichas competencias?
\item ¿Cuáles son las competencias que se han evaluado de forma automática o asistida por ordenador a partir del uso de los entornos virtuales?
\item ¿Qué herramientas o metodologías se utilizan para evaluar competencias mediante el uso de los entornos virtuales?
\item ¿Con qué técnicas se pueden obtener evidencias objetivas del desarrollo de las siguientes competencias en un entorno virtual?).
\item ¿Son utilizados los registros de actividad de los entornos virtuales para la evaluación?
%\item ¿Los resultados que se obtendrían de la evaluación de competencias genéricas de los alumnos mediante el uso de los registros de actividad de los entornos virtuales son indicadores fiables del desempeño de dichas competencias?
\end{enumerate}

\section{Protocolo de revisión}

La definición del protocolo de revisión requiere la realización de una serie de acciones para obtener la bibliografía de nuestro estudio. Comenzaremos indicando los motores de búsqueda que vamos a utilizar, qué términos de búsqueda utilizaremos en dichos motores y las herramientas de soporte a la revisión. Además se mostrarán qué criterios de inclusión de la bibliografía se siguen y el procedimiento de selección.
% Incluir algo de 'Esquema para la extracción de datos'??

\subsection{Motores de búsqueda}
Una vez que las preguntas de la investigación se han establecido, hay que identificar con precisión la estrategia de búsqueda a seguir. En nuestro procedimiento, para encontrar la bibliografía, se realizarán consultas en las siguientes bibliotecas digitales: Wiley Online Library, World Scientific Net, Springer, ACM Digital Library, IEEE Digital Library (Xplore) y Scopus.

\subsection{Términos de búsqueda}
\label{sec:TerminosBusqueda}
Existen muchos términos que pueden utilizarse para referirse a la evaluación de competencias genéricas de manera automatizada o asistida. Por la naturaleza de nuestro trabajo, debemos contemplar siempre en las palabras de búsqueda los términos \emph{assessment} y \emph{generic skills} o \emph{generic competences}. Realizar la búsqueda por el término \emph{Assessment of generic skills} o \emph{assessing generic skills} nos planteaba la primera problemática, y es que el número de artículos devueltos era muy reducido. Por ejemplo, en la \emph{Wiley Online Library} la búsqueda del término exacto \emph{generic skills assessment} devolvió un único resultado. Sin embargo, debilitar la búsqueda con términos como \emph{generic competences} o \emph{generic skills} junto con la palabra \emph{assessment} daba un número de resultados muy elevado. En la misma biblioteca, buscar por los términos \emph{``generic skills`` and student and assessment} nos devolvía 609 resultados. En primera instancia se probó añadiendo términos como  \emph{E-Learning}, \emph{computer-assisted} o \emph{mobile learning}. Sin embargo, incluir términos de este tipo reducían también drásticamente el número de resultados obtenidos en la búsqueda, no llegando a obtenerse bibliografía más significativa que si no se incluyen. Por tanto, a tenor de las pruebas se decide eliminar de la búsqueda ese tipo de términos. Por otro lado, sí se incluyen acrónimos de diferentes entornos virtuales relacionados con las TEL, como son: \emph{TEL}, \emph{LMS}, \emph{ICT} (Information and Communications Technology), \emph{CBI} (Content-Based Instruction). Y tras varias pruebas, se descartan también de la búsqueda términos como `\emph{ICE} (Integrated Collaboration Environment) y \emph{CSCL} (Computer Supported Collaborative Learning), debido a que son términos que en conjunción con los términos principales de nuestra búsqueda no suelen aparecer y los resultados de estas búsquedas eran nulos. Un ejemplo de esto se refleja en una de las consultas realizadas en \emph{Scopus}, dónde los términos \emph{((``student assessment`` OR ``assessment of students``) AND (``generic skills`` OR ``generic competences``)) AND CSCL} no devolvían ningún resultado. La combinación de los términos de búsqueda empleados en la investigación, así como a los motores de búsqueda que fueron aplicados en cada una pueden comprobarse en la tabla \ref{tab:ResumenBusqueda}.

\begin{table}[H]
  \begin{center}
  \begin{tabular}{| m{4cm} | m{7cm} | m{3cm} |}
    \hline
    SOURCE & SEARCH TERMS & SEARCH SCOPE \\
    \hline
    \hline
    Wiley Online Library & assessment AND ``generic competences`` OR ``generic skills`` AND (TEL OR ICT OR CBI) & in All Fields\\
    \hline
    World Scientific Net & ``generic competences`` OR ``generic skills`` AND assessment & Anywhere in article\\
    \hline
    Springer & (``generic skills`` OR ``generic competences``) AND  students AND (TEL OR CBI OR ICT) & All fields (Including full text)\\
    \hline
    ACM Digital Library & (assessment and ``generic skills``) and (TEL or LMS or ICT or CBI) & Any field (title, abstract, review)\\
    \hline
    ACM Digital Library & (assessment and ``generic competences``) and (TEL or LMS or ICT or CBI) & Any field (title, abstract, review)\\
    \hline
    IEEE Digital Library (Xplore) & (((TEL or LMS or ICT or CBI) AND (``generic skills`` OR ``generic competences``)) AND assessment) & Full text and metadata\\
    \hline
    Scopus & (((TEL or LMS or ICT or CBI) AND (``generic skills`` OR ``generic competences``)) AND assessment) & All fields (Including full text)\\
    \hline
  \end{tabular}
\end{center}
\caption{Resumen de búsqueda de bibliografía}
\label{tab:ResumenBusqueda}
\end{table} 

% Incluir tabla con resultados descartados (Supongo que en anexo).

\subsection{Criterios de selección}
Para determinar si un trabajo debía formar parte de nuestra selección de estudios primarios se leyó tanto el título, como el resumen y las palabras clave. En ocasiones esto no fue suficiente, siendo necesario complementar la lectura anterior con una somera la lectura del artículo completo y más detallada de la introducción y las conclusiones.
Nuestra búsqueda se centró en la localización de los trabajos que, habiendo sido obtenidos en el proceso de búsqueda anterior, vayan en línea con nuestro estudio y puedan ayudarnos a resolver las preguntas de investigación. Para ello, se realizó la proyección de los trabajos seleccionados utilizando los siguientes criterios de exclusión:
\begin{itemize}
\item Off Topic: trabajo no relacionado directamente con nuestra investigación. Son trabajos, que aún satisfaciendo los criterios de búsqueda porque de alguna forma se mencionan en el texto, su contribución no está relacionada con la temática de este estudio.
\item Unsupported Language: trabajo escrito en un lenguaje diferente al inglés o español. La mayoría de los textos son en inglés, por lo que este criterio de descarte apenas es utilizado.
\item Duplicated: trabajos cuya contribución principal está recogida en otros trabajos ya incluidos. 
\item Unread: trabajo que no ha podido ser leido. Son textos que no han sido leídos al no estar disponible para su lectura en las bibliotecas digitales a las que se tiene acceso desde la Universidad de Cádiz ni se ha podido encontrar por otros medios (petición por correo a los autores, búsqueda en otros repositorios de Internet, etc).
\end{itemize}

%Para considerar que un trabajo está relacionado directamente con nuestra contribución ha de ayudarnos a resolver alguna de las preguntas de investigación. Por tanto, un trabajo se incluye si:
%\begin{itemize}
%\item Realizan una evaluación automática de competencias genéricas
%\item Realizan una evaluación de competencias genéricas apoyándose en la tecnología, aunque no de manera automática
%\item Realizan una evaluación de alguna competencia específica apoyándose en la tecnología, utilizando alguna técnica cuya aplicación poduera ser de interés para nuestro estudio
%\item Trabajos que abordan la evaluación de competencias genéricas
%\end{itemize}

\subsection{Esquema para la extracción de datos}

Para la extracción de la información se han dividido los trabajos de acuerdo a los siguientes tres aspectos: tipo de investigación, tipo de contribución y ámbito de aplicación de la investigación. A continuación se discute esta clasificación.

\subsubsection{Tipo de investigación}
Esta clasificación hace referencia al tipo de trabajo de investigación llevado a cabo por el/los investigador/es. Existen diferentes enfoques para la clasificación de los trabajos según el tipo investigación que desarrollan. Algunos de estos sistemas de clasificación son los propuestos por Wieringa \cite{Wieringa:2005} y Hevner \cite{Hevner:2004}. Usamos el primero, ya que es el recomendado en el estudio sistemático de mapeo descrito por Petersen \cite{Petersen:2008}.
\begin{itemize}
\item Solución propuesta (\emph{proposal of solution}): se propone una solución para un problema; la solución puede ser innovadora o una extensión significativa de una técnica existente. Los posibles beneficios y la aplicabilidad de la solución se demuestran por un pequeño ejemplo o una buena línea de argumentación.
\item Validación de investigación (\emph{validation research}): las técnicas investigadas son nuevas y todavía no se han aplicado en la práctica. Estas técnicas podrían ser por ejemplo los experimentos, es decir, el trabajo realizado en un laboratorio.
\item Evaluación de la Investigación (\emph{evaluation research}): las técnicas se aplican en la práctica y se lleva a cabo una evaluación de la técnica. Se muestra cómo se implementa la técnica en la práctica (implementación de la solución) y cuáles son las consecuencias de la aplicación en términos de ventajas y desventajas (evaluación de implementación).
\item Artículos de Experiencia (\emph{experience papers}): trabajos que explican qué y cómo algo se ha llevado a cabo en la práctica. Basado en la experiencia personal del autor.
\item Artículos de opinión (\emph{opinion papers}): estos trabajos expresan la opinión personal de alguien acerca de la bondad o viabilidad de una determinada técnica, o cómo se deben realizar las cosas. No se basan en metodologías de trabajo y de investigación relacionadas.
\item Trabajos filosóficos (\emph{philosophical papers}): estos trabajos esbozan una nueva forma de ver las cosas existentes, estructurando el campo en forma de una taxonomía o un marco conceptual.
\end{itemize}

\subsubsection{Tipo de contribución}
En este apartado se clasifican los trabajos según el tipo de contribución que realizan estos al ámbito en el que se desarrollan. Una vez realizado el estudio sistemático de la literatura y habiendo seleccionado los artículos, se realiza una clasificación en base a la aportación de éstos. El uso de algunos términos puede ser confuso, debido a la interpretación que hace el autor del mismo. Algunos de estos términos son framework, modelo, estrategia, proceso, procedimiento, método o metodología. Nuestra clasificación es la siguiente:
\begin{itemize}
\item Modelo (\emph{model}): es una representación de procesos, modelos o sistemas pertenecientes a un supra-sistema, cuyo fin es el análisis de interacción de ellos para mantener una relación flexible que les permita cumplir su función particular y cumplir la función de dicho supra-sistema.
\item Proceso (\emph{process}): contempla aquellos trabajos cuya contribución sea descrita por los autores como una serie de pasos.
\item Herramienta (\emph{tool}): se utiliza para los artículos que presentan un software independiente o una extensión de algún otro programa.
\item Framework (\emph{framework}): aquí se consideran aquellos trabajos que contribuyen con una combinación de los elementos anteriores (es decir, con un modelo, un proceso y una herramienta).
\item Técnica (\emph{technique}): un procedimiento utilizado para llevar a cabo una actividad o tarea específica. Podría venir acompañado de una herramienta de apoyo.
\end{itemize}

\subsubsection{Ámbito de aplicación de la investigación}
Además de los clasificaciones anteriores, es necesario recoger más información acerca los conceptos que representan la contribución de la investigación. Para ello se recoge información sobre el ámbito de la evaluación de competencias sobre el que se aplica cada contribución. Una vez recogida esta información, se agrupan según sus similitudes, quedando finalmente la siguiente clasificación:
\begin{itemize}
\item Resultados de aprendizaje del curso y rúbricas (\emph{Course’s Learning Outcomes (CLO) and rubrics}): los resultados de aprendizaje del curso se evalúan mediante rúbricas o plantillas de evaluación que miden el rendimiento de los alumnos. Esto proporciona al docente un indicador de sus logros de aprendizaje de cada alumno. Las rúbricas pueden estar o no en soporte informático, pero generalmente no aprovechan la tecnología para automatizar tareas.
\item Evaluación entre iguales y autoevaluación (\emph{peer and self eAssessment}): uno de los problemas con los que se encuentran los profesores es la escalabilidad de la tarea de evaluación de competencias cuando el grupo de alumnos es grande. Hay un gran conjunto de trabajos, que aunque se apoyen en la tecnología para realizar alguna actividad, tienen el problema de que la evaluación ha de ser manual. En estos caso, mediante la autoevaluación o evaluación entre iguales los estudiantes se evalúan. De esta manera no sólo descargan de trabajo al profesor haciendo esta evaluación, sino que además se fomenta la capacidad crítica y de análisis del alumno.
\item Aprendizaje basado en juegos (\emph{Game-Based Learning (GBL)}): el aprendizaje basado en juegos se sirve de juegos que están diseñados expresamente para enseñar al usuario acerca de ciertos temas, ampliar conceptos o reforzar el desarrollo o aprendizaje de una habilidad mientras juegan. En ellos los alumnos tienen que completar diferentes pruebas o fases obteniendo puntos en cada una de ellas. Por cada prueba o fase superada, el jugador, o alumno en este caso, obtendrá una serie de puntos. Se podrá decir que un alumno ha alcanzado el nivel de madurez necesario en una competencia si alcanza una predefinida puntuación.
\item E-Evaluación y revisiones (\emph{eAssessment and reviews}): trabajos en los que se obtienen indicadores del desempeño de estudiantes en una o varias competencias de manera automática mediante el uso de algún software. Además se muestran otros trabajos sobre la situación actual en la evaluación de competencias genéricas, su importancia actual y sobre un conjunto de técnicas, metodologías o herramientas que se han desarrollado y utilizado.
\end{itemize}

\subsection{Visualización y análisis de los datos}
Tras obtener los estudios primarios, hay una etapa de análisis, donde se resumen los datos extraídos para así responder a las preguntas de investigación planteadas. El análisis de los resultados se centra en el estudio de las publicaciones para cada categoría y por lo tanto, la determinación del grado de cobertura de cada categoría. Esta información generalmente se resume en tablas y/o gráficos. Otro método utilizado en nuestro estudio es la combinación de diferentes categorías (por ejemplo, el ámbito de investigación contra el tipo contribución) y mostrarlos en un mapa sistemático en la forma de un gráfico de burbujas.
En el siguiente capítulo se mostrarán los resultados obtenidos.

% Conforme redacte el documento quizás haya que completar por aquí

