\section{EvalCourse}
En este trabajo hemos buscado literatura que aborde la evaluación de competencias genéricas, y a ser posible, que lo hagan de manera automática utilizando la interacción de los estudiantes con el LMS. En línea con este trabajo, y a fin de abordar lo que será mi futura tesis doctoral, estamos trabajando en una herramienta que cumpla este propósito. Esta herramienta, cuyo prototipo ya ha sido presentado en varios congresos \cite{Balderas:2013,Balderas:2013a}, se llama EvalCourse. EvalCourse es un Lenguaje de Dominio Específico (DSL). Un DSL es un lenguaje de programación orientado a un problema específico y con una semántica orientada al dominio para el que se diseña. En nuestro caso, este dominio es la evaluación de indicadores de competencias de los estudiantes. El objetivo de EvalCourse es ayudar al docente en la evaluación del desempeño de los alumnos en las competencias que éstos deben desarrollar a lo largo del curso. Estos datos deberán servir al docente como indicadores del desarrollo de dichas competencias. La idea inicial con la que se ha desarrollado este proyecto, es que el DSL sea genérico y pueda ser utilizado con otros sistemas, no sólo de tipo LMS, sino también con sistemas que fomenten el trabajo colaborativo, como pueden ser los wikis. Por ejemplo, MediaWiki es utilizado también en el ámbito de la docencia en nuestra Universidad.

\section{Proyectos en curso}
Para el presente curso se solicitó un proyecto en la convocatoria de actuaciones avaladas para la mejora docente, formación del profesorado y disfusión de resultados, con el título \emph{Extracción de indicadores objetivos para evaluación del desarrollo de competencias genéricas a partir de registros de actividad del Campus Virtual}. La Unidad de Innovación Docente de la Universidad de Cádiz nos concedió dicha actuación y actualmente estamos trabajando en este proyecto.

Este proyecto se engloba dentro del trabajo realizado dentro del grupo de investigación SPI-FM (TIC-195) por parte de Antonio Balderas al uso de las tecnologías en los entornos de aprendizaje. Se pretende desarrollar y aplicar EvalCourse, un DSL que permita a los docentes mediante el uso de una sintaxis sencilla obtener indicadores del desempeño de los estudiantes en diferentes competencias. Son varias las asignaturas de Grado en Ingeniería Informática que participan en este proyecto. Las tareas que comprende son:

\begin{enumerate}
\item Analizar los tipos de indicadores que se podrían obtener de la base de datos Moodle que sean de utilidad para los profesores, así como la manera de obtenerlos.
\item Análisis, desarrollo y prueba del software. Implementar el software DSL para que se conecte a Moodle y pueda extraer la información. Para ello sería necesario la instalación de un entorno de pruebas Moodle y la generación de una batería de datos de pruebas que permita verificar el funcionamiento del software.
\item Obtención de datos reales de las asignaturas del campus virtual. A partir de estos, utilizaremos la herramienta para obtener todos los indicadores necesarios para evaluar las competencias de los estudiantes. Además podremos corregir aquellos fallos que se detecten en el software.
\item Análisis de resultados de la experiencia, difusión del trabajo realizado, publicación de informes, correcciones menores en el sistema, etc. Se analizarán los resultados obtenidos y se darán a conocer el software y la experiencia en webs especializadas, redes sociales, etc. Igualmente se procederá a la correcta documentación del sistema en una web oficial para poder replicar la experiencia (incidiendo en cómo evitar los errores cometidos), así como pequeños desarrollos que corrijan deficiencias observadas y faciliten la reutilización del sistema.
\end{enumerate}

Los objetivos del proyecto son los siguientes:
\begin{enumerate}
\item Realizar un análisis en profundidad de qué indicadores se pueden extraer del campus virtual basado en Moodle que realmente reflejen el desarrollo o no de una competencia por parte de los alumnos.
\item Desarrollar un software basado en un DSL que permita el profesor de manera sencilla y escalable una evaluación objetiva del desarrollo de las competencias por parte de sus alumnos.
\item Aplicar la herramientas los datos extraídos de los cursos del campus virtual de las asignaturas participantes en el proyecto.
\item Análisis de los resultados obtenidos y difusión del trabajo realizado.
\end{enumerate}



