Una vez realizado la localización de los trabajos, la extracción de la información y la clasificación de los estudios en los capítulos anteriores, en esta sección se aborda la respuesta a las preguntas de investigación a tenor de los datos obtenidos en el estudio.

%, a continuación se dan varias ideas que podrían ser extraídas de los resultados del estudio, y después, debatimos algunas de las posibles amenazas a la validez de este trabajo.


%\section{Respuestas a preguntas de investigación}
%En los artículos de este trabajo se tratan diversas competencias. Unas se trabajan y evalúan, otras sólo se mencionan y otros evalúan competencias no reconocidas como tal. En los siguientes apartados, correspondientes a cada una de las preguntas de investigación, se han separado las respuestas por competencias, indicando qué artículos pueden servir como ejemplo para cada caso. 
Las preguntas de investigación recordamos que eran las siguientes:
\begin{enumerate}
\item ¿Cuáles son las competencias que se han evaluado de forma automática o asistida por ordenador a partir del uso de los entornos virtuales?
\item ¿Qué herramientas o metodologías se utilizan para evaluar competencias mediante el uso de los entornos virtuales?
\item ¿Con qué técnicas se pueden obtener evidencias objetivas del desarrollo de las siguientes competencias en un entorno virtual?.
\item ¿Son utilizados los registros de actividad de los entornos virtuales para la evaluación?
\end{enumerate}

Para responder a la primera pregunta: \emph{¿Cuáles son las competencias que se han evaluado de forma automática o asistida por ordenador a partir del uso de los entornos virtuales?}, podemos ver la tabla \ref{tab:CuadroAlias}. En ella se listan las competencias evaluadas en los trabajos recopilados. Además se proporciona un alias, que utilizaremos más adelante para referirnos a cada competencia y no tener que escribir siempre la descripción completa.

%Hay artículos, que hacen una revisión de trabajos como puede ser \cite{Redecker:2012}.

Para responder a las preguntas sobre qué tipo de herramienta (\emph{¿Qué herramientas o metodologías se utilizan para evaluar competencias mediante el uso de los entornos virtuales?}) y con qué técnica \emph{¿Con qué técnicas se pueden obtener evidencias objetivas del desarrollo de las siguientes competencias en un entorno virtual?} se utiliza en cada artículo para cada competencia, se han creado las tablas \ref{tab:CuadroPreguntas1} y \ref{tab:CuadroPreguntas2}. En ellas se indica la herramienta y técnica con que han sido evaluadas las competencias. Hemos utilizado para referirnos a las competencias el alias indicado en la tabla \ref{tab:CuadroAlias}.


\begin{table}[H]
  \begin{center}
  \begin{tabular}{| m{10cm} | l |}
    \hline
    Competencia & Alias \\
    \hline
    \hline 
    Capacidad para comunicarse en un segundo idioma & Idioma \\
    \hline
    Capacidad para comunicarse de forma oral y por medio de la palabra escrita en la lengua materna & Comunicación \\
    \hline 
    Capacidad para generar nuevas ideas (creatividad) & Creatividad\\
    \hline
    Capacidad para trabajar en equipo & Equipo \\
    \hline
    Capacidad para tomar decisiones razonadas & Decisión \\
    \hline
    Capacidad para planificar y administrar el tiempo & Planificación \\
    \hline
    Capacidad para adaptarse y actuar en nuevas situaciones & Adaptación\\
    \hline
    Capacidad para identificar, plantear y resolver problemas & Resolutivo \\
    \hline
    La apreciación y el respeto por la diversidad y multiculturalidad & Multiculturalidad\\
    \hline
    Capacidad de aplicar los conocimientos en situaciones prácticas & Práctico \\
    \hline
    Habilidad para trabajar de forma autónoma & Autonomía \\
    \hline
    Capacidad para el pensamiento abstracto, análisis y síntesis  & Análisis\\
    \hline
    Capacidad para ser críticos y autocríticos & Crítico \\
    \hline
    Espíritu de empresa, la capacidad de tomar la iniciativa & Iniciativa \\
    \hline
  \end{tabular}
\end{center}
\caption{Alias utilizados para las competencias}
\label{tab:CuadroAlias}
\end{table} 

%-----------
% PREGUNTA 4
%-----------
Con respecto a la última pregunta (\emph{¿Son utilizados los registros de actividad de los entornos virtuales para la evaluación?}) no se han encontrado trabajos que aborden esta problemática de manera directa, aunque sí se han encontrado evidencias en la literatura que puedan ayudar a responder a esta pregunta. En alguno de los trabajos se utilizan wikis para fomentar el desempeño de los alumnos en diferentes competencias \cite{Piedra:2010}. Lim indica que los tutores pueden recoger una gran cantidad de información acerca de sus estudiantes mediante la observación de desempeño de los alumnos, mediante la construcción de los wikis, los comentarios realizados por los estudiantes, y los intercambios entre estudiante \cite{Lim:2011}. Pero es evidente que si el curso tiene un gran número de alumnos la observación se vuelve inabordable. Nuestro trabajo va en la línea de observar el trabajo de los alumnos en los LMS mediante la extracción automática de indicadores. Aunque un LMS es más completo en cuanto al ámbito de actuación que un wiki, en ambos los usuarios trabajan en el sistema de forma independiente, dejando un rastro de su actividad y en ambos se sustenta el trabajo colaborativo. Es más, los LMS suelen incluir un wiki en su estructura. 

Por último, Cardona dice que la evaluación de las competencias implica la identificación de los elementos en torno a los procesos de aprendizaje, lo que hace que sea una actividad constante que requiere de criterios para evaluar los resultados durante la formación de las personas \cite{Cardona:2013}. Consideramos que el rastro que dejan los alumnos, su interacción con el entorno, es un reflejo de su trabajo. Chebil indica cómo se podrían analizar las situaciones que se dan en la aplicación de las tecnologías al aprendizaje TEL mediante la recopilación de los rastros de interacción producidos por estos entornos \cite{Chebil:2012}. Mientras que Florian argumenta que para poder utilizar esta información almacenada acerca de las actividades del curso, se requiere filtrado antes de que pueda ser utilizado para procesarla \cite{Florian:2011}.

\pagestyle{apaisado}
\begin{landscape}
\begin{table}[H]
  \begin{center}
  %\begin{tabular}{| c | m{1cm} | m{1cm} | m{1cm} | m{1.5cm} | m{1cm} | m{1cm} | m{1cm} | m{1cm} | m{1cm} | m{1cm} | m{1cm} | m{1cm} | m{1cm} | m{1.5cm} | m{1.4cm} | }
\begin{tabular}{| c | c | c | c | c | c | c | c | c | c | c | c | c | c | c | c | c | c | c |}
    \hline
    	\scriptsize{Competencia} & %1
	\multicolumn{3}{c|}{Idioma} & %2
	\multicolumn{5}{c|}{Comunicación - Creatividad} & %5
	\multicolumn{5}{c|}{Equipo} & %10
	\multicolumn{5}{m{4cm}|}{Decisión - Planificación - Adaptación} \\ %16
    \hline
    %\scriptsize{Herramienta} & \scriptsize{Facebook} & \scriptsize{Second Life} & \scriptsize{Grabación en vídeo} & \scriptsize{Wiki} & -- & \scriptsize{Grabación en vídeo} & -- & \scriptsize{Grabación en vídeo} & \multicolumn{2}{m{2cm}|}{\scriptsize{Wiki}} & \multicolumn{5}{ c|}{\scriptsize{Otros}} \\
\scriptsize{Herramienta} & Facebook & \multicolumn{1}{m{1.2cm}|}{Second Life} & \multicolumn{1}{m{1.4cm}|}{Grabación en vídeo} & GBL & \multicolumn{1}{m{0.6cm}|}{Wiki} & \multicolumn{1}{m{0.7cm}|}{Otros} & \multicolumn{1}{m{1.4cm}|}{Grabación en vídeo} & \multicolumn{1}{m{0.7cm}|}{Otros} & \multicolumn{1}{m{1.4cm}|}{Grabación en vídeo} & \multicolumn{2}{m{1.3cm}|}{Wiki - Portfolio} & \multicolumn{6}{ c|}{Otros} & GBL \\
    \hline
    \scriptsize{Técnica} & \multicolumn{3}{m{2.5cm}|}{Autoevaluación y/o evaluación entre iguales} & \multicolumn{1}{m{1.8cm}|}{eAssessment} & \multicolumn{2}{m{0.9cm}|}{Rúbricas} & \multicolumn{4}{m{2.5cm}|}{Autoevaluación y/o evaluación entre iguales} & \multicolumn{2}{m{1cm}|}{Rúbricas} & \multicolumn{4}{m{2.5cm}|}{Autoevaluación y/o evaluación entre iguales} & \multicolumn{1}{m{1.1cm}|}{Rúbricas} & \multicolumn{1}{m{1.8cm}|}{eAssessment} \\
    \hline
    \hline 
    \cite{Shih:2011} 			& X  & -- & --  & -- & -- & -- & -- & -- & --  & -- & -- & -- & -- & -- & -- & -- & -- & -- \\
    \hline
    \cite{MercedesRico:2013} 		& -- & X  & --  & -- & -- & -- & -- & --  & -- & -- & -- & -- & -- & -- & -- & -- & -- & -- \\
    \hline
    \cite{Masip-Alvarez:2013} 		& -- & -- & X   & -- & -- & --  & X  & -- & X  & -- & -- & -- & -- & -- & -- & -- & -- & -- \\
    \hline
    \cite{Mohamed:2008} 		& -- & -- & --  & -- & --  & X  & -- & -- & -- & -- & -- & -- & -- & -- & -- & -- & -- & -- \\
    \hline
    \cite{Piedra:2010} 			& -- & -- & --  & -- & X  & -- & -- & -- & -- & X  & -- & -- & -- & -- & --  & -- & -- & -- \\
    \hline
    \cite{Liao:2013} 			& -- & -- & --  & -- & -- & -- & -- & X   & -- & -- & -- & -- & X  & -- & X & X & -- & -- \\
    \hline
    \cite{Colomo-Palacios:2013} 	& -- & -- & --  & -- & -- & -- & -- & X   & -- & -- & -- & -- & X  & X & -- & -- & -- & -- \\
    \hline
    \cite{McMahon:2007} 		& -- & -- & --   & -- & -- & -- & -- & --  & -- & -- & -- & -- & X  & -- & -- & -- & -- & -- \\
    \hline
    \cite{Rashid:2008} 			& -- & -- & --   & -- & -- & -- & -- & --  & -- &  -- & -- & X & -- & -- & -- & --  & -- & -- \\
    \hline
    \cite{Lim:2011} 			& -- & -- & --   & -- & -- & -- & -- & --  & -- & -- & X & --  & -- & -- & -- & -- & -- & -- \\
    \hline
    \cite{Achcaoucaou:2012} 		& -- & -- & --   & -- & -- & -- & --  & -- & -- & -- & -- & -- & -- & X & X & --  & -- & -- \\
    \hline
    \cite{Guenaga:2013} 		& -- & -- & --   & -- & -- & -- & -- & -- & -- & --  & -- & -- & -- & -- & -- & --  & -- & -- \\
    \hline
    \cite{Martin-Cuadrado:2013} 	& -- & -- & --   & -- & -- & --& -- & --  & X  & -- & -- & -- & -- & -- & -- & -- & -- & -- \\
    \hline
    \cite{Mohamed:2008a} 		& -- & -- & --   & -- & -- & -- & -- & -- & -- & --  & -- & -- & -- & -- & -- & -- & -- & --  \\
    \hline
    \cite{Gil:2011} 			& -- & -- & --   & -- & -- & -- & -- & --  & -- & -- & X & --  & -- & -- & -- & -- & -- & -- \\
    \hline
    \cite{Velasco:2012}			& -- & -- & --   & -- & -- & -- & -- & --  & -- & -- & X & --  & -- & -- & -- & -- & X  & -- \\
    \hline
    \cite{Borrajo:2010}			& -- & -- & --   & -- & -- & -- & -- & --  & -- & -- & -- & --  & -- & -- & -- & --  & -- & X \\
    \hline
    \cite{Bedek:2011}			& -- & -- & --   & X & -- & -- & -- & --  & -- & -- & -- & --  & -- & -- & -- & --  & -- & -- \\
    \hline
    \cite{Palomares:2011}		& -- & -- & --   & -- & -- & -- & -- & --  & -- & -- & X & --  & -- & -- & -- & -- & -- & -- \\
    \hline
  \end{tabular}
\end{center}
\caption{Primer cuadro de competencias - herramientas - técnicas}
\label{tab:CuadroPreguntas1}
\end{table} 

\begin{table}[H]
  \begin{center}
  \begin{tabular}{| c | c | c | c | c | c | c | c | c | c | c | c |}
    \hline
    \scriptsize{Competencia} & \multicolumn{3}{c|}{Resolutivo} & Multiculturalidad & Práctico & \multicolumn{2}{c|}{Autonomía} & Análisis & Crítico &  \multicolumn{2}{c|}{Iniciativa} \\
    \hline
    \scriptsize{Herramienta} & GBL & \multicolumn{5}{c|}{Otros} & Portfolio & \multicolumn{3}{c|}{Otros} & GBL \\
    \hline
    \scriptsize{Técnica} & \multicolumn{2}{c|}{Rúbrica} & \multicolumn{4}{c|}{Autoevaluación y/o evaluación entre iguales} & Rúbricas & \multicolumn{3}{c|}{Autoevaluación y/o evaluación entre iguales} & eAssessment \\
    \hline
    \hline 
    \cite{Shih:2011} 		& -- & -- & -- & -- & -- & -- & -- & -- & -- & -- & -- \\
    \hline
    \cite{MercedesRico:2013} 	& -- & -- & -- & -- & -- & -- & -- & -- & -- &  -- & -- \\
    \hline
    \cite{Masip-Alvarez:2013} 	& -- & -- & -- & -- & -- & -- & -- & -- & -- &  -- & --\\
    \hline
    \cite{Mohamed:2008} 	& -- & -- & -- & -- & -- & -- & -- & -- & -- &  -- & --\\
    \hline
    \cite{Piedra:2010} 		& -- & -- & -- & -- & -- & -- & -- & -- & -- & -- & -- \\
    \hline
    \cite{Liao:2013} 		& -- & -- & -- & X  & X  & -- & -- & -- & -- & -- & -- \\
    \hline
    \cite{Colomo-Palacios:2013} & -- & -- &  X &  X & -- & X  & -- & X & X & X & -- \\ % 7 y 11
    \hline
    \cite{McMahon:2007} 	& -- & -- & -- & -- & -- & -- & -- & -- & -- & -- & -- \\
    \hline
    \cite{Rashid:2008} 		& -- & -- & -- & -- & -- & -- & -- & -- & -- &  -- & -- \\
    \hline
    \cite{Lim:2011} 		& -- & -- & -- & -- & -- & -- & -- & -- & -- & -- & -- \\
    \hline
    \cite{Achcaoucaou:2012} 	& -- & -- & X & -- & -- & -- & -- & -- & -- &  -- & -- \\
    \hline
    \cite{Guenaga:2013} 	& X & -- & -- & -- & -- & -- & -- & -- &  -- & -- & -- \\
    \hline
    \cite{Martin-Cuadrado:2013} & -- & -- & -- & -- & -- & -- & -- & -- &  -- & -- & -- \\
    \hline
    \cite{Mohamed:2008a} 	& -- & X  & -- & -- & -- & -- & -- & -- & -- &  -- & -- \\
    \hline
    \cite{Gil:2011} 		& -- & --  & -- & -- & -- & -- & -- & -- & -- &  -- & -- \\
    \hline
    \cite{Velasco:2012}		& -- & --  & -- & -- & -- & -- & -- & -- & -- &  -- & -- \\
    \hline
    \cite{Borrajo:2010}		& -- & --  & -- & -- & -- & -- & -- & -- & --  & -- &  X \\
    \hline
    \cite{Bedek:2011}		& -- & --  & -- & -- & -- & -- & -- & -- & --  & -- &  -- \\
    \hline
    \cite{Palomares:2011}	& -- & --  & -- & -- & -- & -- & X  & -- & --  & -- &  -- \\
    \hline
  \end{tabular}
\end{center}
\caption{Segundo cuadro de competencias - herramientas - técnicas}
\label{tab:CuadroPreguntas2}
\end{table} 
\end{landscape}

\pagestyle{normal}




