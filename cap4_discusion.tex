Una vez realizado la localización de los trabajos, la extracción de la información y la clasificación de los estudios en los capítulos anteriores, en esta sección se aborda la respuesta a las preguntas de investigación a tenor de los datos obtenidos en el estudio.

%, a continuación se dan varias ideas que podrían ser extraídas de los resultados del estudio, y después, debatimos algunas de las posibles amenazas a la validez de este trabajo.


%\section{Respuestas a preguntas de investigación}
En los artículos de este trabajo se tratan diversas competencias. Unas se trabajan y evalúan, otras sólo se mencionan y otros evalúan competencias no reconocidas como tal. En los siguientes apartados, correspondientes a cada una de las preguntas de investigación, se han separado las respuestas por competencias, indicando qué artículos pueden servir como ejemplo para cada caso. 

Para tener una visión resumida de a qué tipo de herramienta, con qué técnica se utiliza en cada artículo para cada competencia, se han creado las tablas \ref{tab:CuadroPreguntas1} y \ref{tab:CuadroPreguntas2}. Para que ocupe menos espacio se ha asignado un alias a cada competencias que puede consultar en la tabla \ref{tab:CuadroAlias}

%Hay artículos, que hacen una revisión de trabajos como puede ser \cite{Redecker:2012}.

\begin{table}[H]
  \begin{center}
  \begin{tabular}{| m{10cm} | l |}
    \hline
    Competencia & Alias \\
    \hline
    \hline 
    Capacidad para comunicarse en un segundo idioma & Idioma \\
    \hline
    Capacidad para comunicarse de forma oral y por medio de la palabra escrita en la lengua materna & Comunicación \\
    \hline 
    Capacidad para generar nuevas ideas (creatividad) & Creatividad\\
    \hline
    Capacidad para trabajar en equipo & Equipo \\
    \hline
    Capacidad para tomar decisiones razonadas & Decisión \\
    \hline
    Capacidad para planificar y administrar el tiempo & Planificación \\
    \hline
    Capacidad para adaptarse y actuar en nuevas situaciones & Adaptación\\
    \hline
    Capacidad para identificar, plantear y resolver problemas & Resolutivo \\
    \hline
    La apreciación y el respeto por la diversidad y multiculturalidad & Multiculturalidad\\
    \hline
    Capacidad de aplicar los conocimientos en situaciones prácticas & Práctico \\
    \hline
    Habilidad para trabajar de forma autónoma & Autonomía \\
    \hline
    Capacidad para el pensamiento abstracto, análisis y síntesis  & Análisis\\
    \hline
    Capacidad para ser críticos y autocríticos & Crítico \\
    \hline
    Espíritu de empresa, la capacidad de tomar la iniciativa & Iniciativa \\
    \hline
  \end{tabular}
\end{center}
\caption{Alias utilizados para las competencias}
\label{tab:CuadroAlias}
\end{table} 


\pagestyle{apaisado}
\begin{landscape}
\begin{table}[H]
  \begin{center}
  %\begin{tabular}{| c | m{1cm} | m{1cm} | m{1cm} | m{1.5cm} | m{1cm} | m{1cm} | m{1cm} | m{1cm} | m{1cm} | m{1cm} | m{1cm} | m{1cm} | m{1cm} | m{1.5cm} | m{1.4cm} | }
\begin{tabular}{| c | c | c | c | c | c | c | c | c | c | c | c | c | c | c | c | }
    \hline
    	\scriptsize{Competencia} & %1
	\multicolumn{3}{c|}{Idioma} & %2
	\multicolumn{4}{c|}{Comunicación - Creatividad} & %5
	\multicolumn{5}{c|}{Equipo} & %10
	\multicolumn{3}{m{4cm}|}{Decisión - Planificación - Adaptación} \\ %16
    \hline
    %\scriptsize{Herramienta} & \scriptsize{Facebook} & \scriptsize{Second Life} & \scriptsize{Grabación en vídeo} & \scriptsize{Wiki} & -- & \scriptsize{Grabación en vídeo} & -- & \scriptsize{Grabación en vídeo} & \multicolumn{2}{m{2cm}|}{\scriptsize{Wiki}} & \multicolumn{5}{ c|}{\scriptsize{Otros}} \\
\scriptsize{Herramienta} & Facebook & \multicolumn{1}{m{1.5cm}|}{Second Life} & \multicolumn{1}{m{1.5cm}|}{Grabación en vídeo} & Wiki & Otros & \multicolumn{1}{m{1.5cm}|}{Grabación en vídeo} & Otros & \multicolumn{1}{m{1.5cm}|}{Grabación en vídeo} & \multicolumn{2}{m{2cm}|}{Wiki} & \multicolumn{5}{ c|}{Otros} \\
    \hline
    \scriptsize{Técnica} & \multicolumn{3}{m{5cm}|}{Autoevaluación y/o evaluación entre iguales} & \multicolumn{2}{c|}{Rúbricas} & \multicolumn{4}{m{5cm}|}{Autoevaluación y/o evaluación entre iguales} & \multicolumn{2}{c|}{Rúbricas} & \multicolumn{4}{m{5cm}|}{Autoevaluación y/o evaluación entre iguales} \\
    \hline
    \hline 
    \cite{Shih:2011} 			& X  & -- & -- & -- & -- & -- & -- & --  & -- & -- & -- & -- & -- & -- & -- \\
    \hline
    \cite{MercedesRico:2013} 		& -- & X  & -- & -- & -- & -- & --  & -- & -- & -- & -- & -- & -- & -- & -- \\
    \hline
    \cite{Masip-Alvarez:2013} 		& -- & -- & X  & -- & --  & X  & -- & X  & -- & -- & -- & -- & -- & -- & -- \\
    \hline
    \cite{Mohamed:2008} 		& -- & -- & -- & --  & X  & -- & -- & -- & -- & -- & -- & -- & -- & -- & -- \\
    \hline
    \cite{Piedra:2010} 			& -- & -- & -- & X  & -- & -- & -- & -- & X  & -- & -- & -- & -- & --  & -- \\
    \hline
    \cite{Liao:2013} 			& -- & -- & -- & -- & -- & -- & X   & -- & -- & -- & -- & X  & -- & X & X \\
    \hline
    \cite{Colomo-Palacios:2013} 	& -- & -- & -- & -- & -- & -- & X   & -- & -- & -- & -- & X  & X & -- & -- \\
    \hline
    \cite{McMahon:2007} 		& -- & -- & --  & -- & -- & -- & --  & -- & -- & -- & -- & X  & -- & -- & -- \\
    \hline
    \cite{Rashid:2008} 			& -- & -- & --  & -- & -- & -- & --  & -- &  -- & -- & X & -- & -- & -- & --  \\
    \hline
    \cite{Lim:2011} 			& -- & -- & --  & -- & -- & -- & --  & -- & -- & X & --  & -- & -- & -- & -- \\
    \hline
    \cite{Achcaoucaou:2012} 		& -- & -- & --  & -- & -- & --  & -- & -- & -- & -- & -- & -- & X & X & --  \\
    \hline
    \cite{Guenaga:2013} 		& -- & -- & --  & -- & -- & -- & -- & -- & --  & -- & -- & -- & -- & -- & --  \\
    \hline
    \cite{Martin-Cuadrado:2013} 	& -- & -- & --  & -- & --& -- & --  & X  & -- & -- & -- & -- & -- & -- & -- \\
    \hline
  \end{tabular}
\end{center}
\caption{Primer cuadro de competencias - herramientas - técnicas}
\label{tab:CuadroPreguntas1}
\end{table} 

\begin{table}[H]
  \begin{center}
  \begin{tabular}{| c | c | c | c | c | c | c | c | c |}
    \hline
    \scriptsize{Competencia} & \multicolumn{2}{c|}{Resolutivo} & Multiculturalidad & Práctico & Autonomía & Análisis & Crítico & Iniciativa \\
    \hline
    \scriptsize{Herramienta} & Juego Serio & \multicolumn{7}{c|}{Otros} \\
    \hline
    \scriptsize{Técnica} & Rúbrica & \multicolumn{7}{c|}{Autoevaluación y/o evaluación entre iguales} \\
    \hline
    \hline 
    \cite{Shih:2011} & -- & -- & -- & -- & -- & -- & -- & \\
    \hline
    \cite{MercedesRico:2013} & -- & -- & -- & -- & -- & -- & -- &  -- \\
    \hline
    \cite{Masip-Alvarez:2013} & -- & -- & -- & -- & -- & -- & -- &  --\\
    \hline
    \cite{Mohamed:2008} & -- & -- & -- & -- & -- & -- & -- &  --\\
    \hline
    \cite{Piedra:2010} & -- & -- & -- & -- & -- & -- & -- & -- \\
    \hline
    \cite{Liao:2013} & -- & -- & X & X & -- & -- & -- & -- \\
    \hline
    \cite{Colomo-Palacios:2013} & -- & X & X & -- & X & X & X & X \\ % 7 y 11
    \hline
    \cite{McMahon:2007} & -- & -- & -- & -- & -- & -- & -- & -- \\
    \hline
    \cite{Rashid:2008} & -- & -- & -- & -- & -- & -- & -- &  --\\
    \hline
    \cite{Lim:2011} & -- & -- & -- & -- & -- & -- & -- & -- \\
    \hline
    \cite{Achcaoucaou:2012} & --  & X & -- & -- & -- & -- & -- &  --\\
    \hline
    \cite{Guenaga:2013} & X & -- & -- & -- & -- & -- &  -- & --\\
    \hline
    \cite{Martin-Cuadrado:2013} & -- & -- & -- & -- & -- & -- &  -- & -- \\
    \hline
  \end{tabular}
\end{center}
\caption{Segundo cuadro de competencias - herramientas - técnicas}
\label{tab:CuadroPreguntas2}
\end{table} 
\end{landscape}

\pagestyle{normal}
%-----------
% PREGUNTA 1
%-----------
\section{¿Qué competencias se han evaluado de forma automática o asistida por ordenador a partir del uso de los entornos virtuales?}
Las competencias que se han evaluado en los artículos de este trabajo son las siguientes:
\begin{itemize}
\item Capacidad para comunicarse en un segundo idioma \cite{Shih:2011}, \cite{MercedesRico:2013}, \cite{Masip-Alvarez:2013}
\item Capacidad para comunicarse de forma oral y por medio de la palabra escrita en la lengua materna \cite{Mohamed:2008}, \cite{Piedra:2010}, \cite{Liao:2013}, \cite{Masip-Alvarez:2013}, \cite{Colomo-Palacios:2013}
\item Capacidad para generar nuevas ideas (creatividad) \cite{Piedra:2010}, \cite{Liao:2013}, \cite{Colomo-Palacios:2013}
\item Capacidad para trabajar en equipo \cite{McMahon:2007}, \cite{Mohamed:2008}, \cite{Rashid:2008}, \cite{Lim:2011}, \cite{Liao:2013}, \cite{Masip-Alvarez:2013}, \cite{Colomo-Palacios:2013}
\item Capacidad para tomar decisiones razonadas \cite{Achcaoucaou:2012}, \cite{Colomo-Palacios:2013}
\item Capacidad para planificar y administrar el tiempo \cite{Achcaoucaou:2012}, \cite{Liao:2013}
\item Capacidad para adaptarse y actuar en nuevas situaciones \cite{Liao:2013}
\item La apreciación y el respeto por la diversidad y multiculturalidad \cite{Liao:2013}, \cite{Colomo-Palacios:2013}
\item Capacidad de aplicar los conocimientos en situaciones prácticas \cite{Liao:2013}
\item Capacidad para identificar, plantear y resolver problemas \cite{Achcaoucaou:2012}, \cite{Guenaga:2013}, \cite{Colomo-Palacios:2013}
\item Habilidad para trabajar de forma autónoma \cite{Colomo-Palacios:2013}
\item Capacidad para el pensamiento abstracto, análisis y síntesis \cite{Colomo-Palacios:2013}
\item Capacidad para ser críticos y autocríticos \cite{Colomo-Palacios:2013}
\item Espíritu de empresa, la capacidad de tomar la iniciativa \cite{Colomo-Palacios:2013}
\end{itemize}

%-----------
% PREGUNTA 2
%-----------
\section{¿Qué herramientas o metodologías se utilizan para evaluar competencias mediante el uso de los entornos virtuales?}
Las herramientas o metodologías que se han empleado en nuestra selección primaria para cada competencia son las siguientes:
\begin{itemize}
\item Capacidad para comunicarse en un segundo idioma:
	\begin{itemize} 
	\item Facebook \cite{Shih:2011}
	\item Second Life \cite{MercedesRico:2013}
	\item Grabación en vídeo \cite{Masip-Alvarez:2013}
	\end{itemize}
\item Capacidad para comunicarse de forma oral y por medio de la palabra escrita en la lengua materna:
	\begin{itemize} 
	\item Grabación en vídeo \cite{Masip-Alvarez:2013}
	\end{itemize}
\item Capacidad para trabajar en equipo:
	\begin{itemize} 
	\item Entorno de seguimiento de trabajo en equipo. JAMTART (Joe And Marks Tracking And Reporting Tool) \cite{McMahon:2007}
	\item Grabación en vídeo \cite{Masip-Alvarez:2013}, \cite{Martin-Cuadrado:2013}
	\item Wikis \cite{Piedra:2010}, \cite{Lim:2011}
	\item DSWC \cite{Liao:2013}
	\end{itemize}
\item Capacidad para tomar decisiones razonadas:
	\begin{itemize} 
	\item Tricuspoid \cite{Achcaoucaou:2012}, se trata de una plataforma digital diseñada específicamente para la auto-evaluación de competencias empresariales.
	\end{itemize}
\item Capacidad para planificar y administrar el tiempo
	\begin{itemize} 
	\item Tricuspoid \cite{Achcaoucaou:2012}.
	\end{itemize}
\item Capacidad para generar nuevas ideas (creatividad):
	\begin{itemize} 
	\item Wikis, blogs \cite{Piedra:2010}
	\end{itemize}
\item Capacidad para identificar, plantear y resolver problemas:
	\begin{itemize} 
	\item Tricuspoid \cite{Achcaoucaou:2012}.
	\item Juego serio \cite{Guenaga:2013}
	\end{itemize}
\end{itemize}

%-----------
% PREGUNTA 3
%-----------
\section{¿Con qué técnicas se pueden obtener evidencias objetivas del desarrollo de las siguientes competencias en un entorno virtual?}
\begin{itemize}
\item Capacidad para comunicarse en un segundo idioma:
	\begin{itemize} 
	\item Evaluación entre iguales y/o autoevaluación \cite{Shih:2011}, \cite{Masip-Alvarez:2013}
	\item Evaluación conjunta de varios profesores \cite{MercedesRico:2013}
	\end{itemize}
\item Capacidad para comunicarse de forma oral y por medio de la palabra escrita en la lengua materna
	\begin{itemize} 
	\item Autoevaluación \cite{Liao:2013}
	\item Evaluación entre iguales y/o autoevaluación \cite{Masip-Alvarez:2013}
	\item Rúbricas \cite{Mohamed:2008}
	\end{itemize}
\item Capacidad para generar nuevas ideas (creatividad)
	\begin{itemize} 
	\item Autoevaluación \cite{Liao:2013}
	\end{itemize}
\item Capacidad para trabajar en equipo
	\begin{itemize} 
	\item Autoevaluación y/o evaluación entre iguales \cite{McMahon:2007}, \cite{Lim:2011}, \cite{Masip-Alvarez:2013}, \cite{Liao:2013}
	\item Rúbricas \cite{Mohamed:2008}, \cite{Piedra:2010}
	\item Técnica estadística: Modelo ESPEGS \cite{Rashid:2008}
	\end{itemize}
\item Capacidad para tomar decisiones razonadas:
	\begin{itemize} 
	\item Autoevaluación y/o evaluación entre iguales \cite{Achcaoucaou:2012}
	\end{itemize}
\item Capacidad para planificar y administrar el tiempo
	\begin{itemize} 
	\item Autoevaluación y/o evaluación entre iguales \cite{Achcaoucaou:2012}, \cite{Liao:2013}
	\end{itemize}
\item Capacidad para adaptarse y actuar en nuevas situaciones
	\begin{itemize} 
	\item Autoevaluación \cite{Liao:2013}
	\end{itemize}
\item La apreciación y el respeto por la diversidad y multiculturalidad:
	\begin{itemize} 
	\item Autoevaluación \cite{Liao:2013}
	\end{itemize}
\item Capacidad de aplicar los conocimientos en situaciones prácticas:
	\begin{itemize} 
	\item Autoevaluación \cite{Liao:2013}
	\end{itemize}
\item Capacidad para identificar, plantear y resolver problemas:
	\begin{itemize} 
	\item Autoevaluación y/o evaluación entre iguales \cite{Achcaoucaou:2012}
	\item Indicadores basados en rúbrica \cite{Guenaga:2013}
	\end{itemize}
\item Habilidad para trabajar de forma autónoma:
	\begin{itemize} 
	\item Autoevaluación y/o evaluación entre iguales \cite{Colomo-Palacios:2013}
	\end{itemize} 
\item Habilidad para trabajar de forma autónoma:
	\begin{itemize} 
	\item Autoevaluación y/o evaluación entre iguales \cite{Colomo-Palacios:2013}
	\end{itemize} 
\item Capacidad para el pensamiento abstracto, análisis y síntesis:
	\begin{itemize} 
	\item Autoevaluación y/o evaluación entre iguales \cite{Colomo-Palacios:2013}
	\end{itemize} 
\item Capacidad para ser críticos y autocríticos:
	\begin{itemize} 
	\item Autoevaluación y/o evaluación entre iguales \cite{Colomo-Palacios:2013}
	\end{itemize} 
\item Espíritu de empresa, la capacidad de tomar la iniciativa:
	\begin{itemize} 
	\item Autoevaluación y/o evaluación entre iguales \cite{Colomo-Palacios:2013}
	\end{itemize} 
\end{itemize}

%-----------
% PREGUNTA 4
%-----------
\section{¿Son utilizados los registros de actividad de los entornos virtuales para la evaluación?}

No se han encontrado trabajos que aborden esta problemática de manera directa, aunque sí se han encontrado evidencias en la literatura que puedan ayudar a responder a esta pregunta. En alguno de los trabajos se utilizan wikis para fomentar el desempeño de los alumnos en diferentes competencias \cite{Piedra:2010}. Lim indica que los tutores pueden recoger una gran cantidad de información acerca de sus estudiantes mediante la observación de desempeño de los alumnos, mediante la construcción de los wikis, los comentarios realizados por los estudiantes, y los intercambios entre estudiante \cite{Lim:2011}. Pero es evidente que si el curso tiene un gran número de alumnos la observación se vuelve inabordable. Nuestro trabajo va en la línea de observar el trabajo de los alumnos en los LMS mediante la extracción automática de indicadores. Aunque un LMS es más completo en cuánto al ámbito de actuación que un wiki, en ambos los usuarios trabajan en el sistema de forma independiente, dejando un rastro de su actividad y en ambos se sustenta el trabajo colaborativo. Es más, los LMS suelen incluir un wiki en su estructura. 

Por último, Cardona dice que la evaluación de las competencias implica la identificación de los elementos en torno a los procesos de aprendizaje, lo que hace que sea una actividad constante que requiere de criterios para evaluar los resultados durante la formación de las personas \cite{Cardona:2013}. Consideramos que el rastro que dejan los alumnos, su interacción con el entorno, es un reflejo de su trabajo. Chebil indica como se podrían analizar las situaciones que se dan en la aplicación de las tecnologías al aprendizaje TEL mediante la recopilación de los rastros de interacción producidos por estos entornos \cite{Chebil:2012}. Mientras que Florian argumenta que para poder utilizar esta información almacenada acerca de las actividades del curso, se requiere filtrado antes de que pueda ser utilizado para procesarla \cite{Florian:2011}.


