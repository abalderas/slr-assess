En esta sección, se trata de dar respuesta a las preguntas de investigación.

%, a continuación se dan varias ideas que podrían ser extraídas de los resultados del estudio, y después, debatimos algunas de las posibles amenazas a la validez de este trabajo.


\section{Respuestas a preguntas de investigación}
En los artículos de este trabajo se tratan diversas competencias. Unas se trabajan y evalúan, otras sólo se mencionan y otros evalúan competencias no reconocidas como tal. En los siguientes apartados, correspondientes a cada una de las preguntas de investigación, se han separado las respuestas por competencias, indicando qué artículos pueden servir como ejemplo para cada caso. %Hay artículos, que hacen una revisión de trabajos como puede ser \cite{Redecker:2012}.

%-----------
% PREGUNTA 1
%-----------
\subsection{¿Qué competencias se han evaluado de forma automática o asistida por ordenador a partir del uso de los entornos virtuales?}
Las competencias que se han evaluado en los artículos de este trabajo son las siguientes:
\begin{itemize}
\item Capacidad para comunicarse en un segundo idioma \cite{Shih:2011}, \cite{MercedesRico:2013}, \cite{Masip-Alvarez:2013}
\item Capacidad para comunicarse de forma oral y por medio de la palabra escrita en la lengua materna \cite{Mohamed:2008}, \cite{Piedra:2010}, \cite{Liao:2013}, \cite{Masip-Alvarez:2013}, \cite{Colomo-Palacios:2013}
\item Capacidad para generar nuevas ideas (creatividad) \cite{Piedra:2010}, \cite{Liao:2013}, \cite{Colomo-Palacios:2013}
\item Capacidad para trabajar en equipo \cite{McMahon:2007}, \cite{Mohamed:2008}, \cite{Rashid:2008}, \cite{Lim:2011}, \cite{Liao:2013}, \cite{Masip-Alvarez:2013}, \cite{Colomo-Palacios:2013}
\item Capacidad para tomar decisiones razonadas \cite{Achcaoucaou:2012}, \cite{Colomo-Palacios:2013}
\item Capacidad para planificar y administrar el tiempo \cite{Achcaoucaou:2012}, \cite{Liao:2013}
\item Capacidad para adaptarse y actuar en nuevas situaciones \cite{Liao:2013}
\item La apreciación y el respeto por la diversidad y multiculturalidad \cite{Liao:2013}, \cite{Colomo-Palacios:2013}
\item Capacidad de aplicar los conocimientos en situaciones prácticas \cite{Liao:2013}
\item Capacidad para identificar, plantear y resolver problemas \cite{Achcaoucaou:2012}, \cite{Guenaga:2013}, \cite{Colomo-Palacios:2013}
\item Habilidad para trabajar de forma autónoma \cite{Colomo-Palacios:2013}
\item Capacidad para el pensamiento abstracto, análisis y síntesis \cite{Colomo-Palacios:2013}
\item Capacidad para ser críticos y autocríticos \cite{Colomo-Palacios:2013}
\item Espíritu de empresa, la capacidad de tomar la iniciativa \cite{Colomo-Palacios:2013}
\end{itemize}

%-----------
% PREGUNTA 2
%-----------
\subsection{¿Qué herramientas o metodologías se utilizan para evaluar competencias mediante el uso de los entornos virtuales?}
Las herramientas o metodologías que se han empleado en nuestra selección primaria para cada competencia son las siguientes:
\begin{itemize}
\item Capacidad para comunicarse en un segundo idioma:
	\begin{itemize} 
	\item Facebook \cite{Shih:2011}
	\item Second Life \cite{MercedesRico:2013}
	\item Grabación en vídeo \cite{Masip-Alvarez:2013}
	\end{itemize}
\item Capacidad para comunicarse de forma oral y por medio de la palabra escrita en la lengua materna:
	\begin{itemize} 
	\item Grabación en vídeo \cite{Masip-Alvarez:2013}
	\end{itemize}
\item Capacidad para trabajar en equipo:
	\begin{itemize} 
	\item Entorno de seguimiento de trabajo en equipo. JAMTART (Joe And Marks Tracking And Reporting Tool) \cite{McMahon:2007}
	\item Grabación en vídeo \cite{Masip-Alvarez:2013}, \cite{Martin-Cuadrado:2013}
	\item Wikis \cite{Piedra:2010}, \cite{Lim:2011}
	\end{itemize}
\item Capacidad para tomar decisiones razonadas:
	\begin{itemize} 
	\item Tricuspoid \cite{Achcaoucaou:2012}, se trata de una plataforma digital diseñada específicamente para la auto-evaluación de competencias empresariales.
	\end{itemize}
\item Capacidad para planificar y administrar el tiempo
	\begin{itemize} 
	\item Tricuspoid \cite{Achcaoucaou:2012}.
	\end{itemize}
\item Capacidad para generar nuevas ideas (creatividad):
	\begin{itemize} 
	\item Wikis, blogs \cite{Piedra:2010}
	\end{itemize}
\item Capacidad para identificar, plantear y resolver problemas:
	\begin{itemize} 
	\item Tricuspoid \cite{Achcaoucaou:2012}.
	\item Juego serio \cite{Guenaga:2013}
	\end{itemize}
\end{itemize}

%-----------
% PREGUNTA 3
%-----------
\subsection{¿Con qué técnicas se pueden obtener evidencias (numéricas) objetivas del desarrollo de las siguientes competencias en un entorno virtual?}
\begin{itemize}
\item Capacidad para comunicarse en un segundo idioma:
	\begin{itemize} 
	\item Evaluación entre iguales y/o autoevaluación \cite{Shih:2011}, \cite{Masip-Alvarez:2013}
	\item Evaluación conjunta de varios profesores \cite{MercedesRico:2013}
	\end{itemize}
\item Capacidad para comunicarse de forma oral y por medio de la palabra escrita en la lengua materna
	\begin{itemize} 
	\item Autoevaluación \cite{Liao:2013}
	\item Evaluación entre iguales y/o autoevaluación \cite{Masip-Alvarez:2013}
	\item Rúbricas \cite{Mohamed:2008}
	\end{itemize}
\item Capacidad para generar nuevas ideas (creatividad)
	\begin{itemize} 
	\item Autoevaluación \cite{Liao:2013}
	\end{itemize}
\item Capacidad para trabajar en equipo
	\begin{itemize} 
	\item Autoevaluación y/o evaluación entre iguales \cite{McMahon:2007}, \cite{Lim:2011}, \cite{Masip-Alvarez:2013}, \cite{Liao:2013}
	\item Rúbricas \cite{Mohamed:2008}, \cite{Piedra:2010}
	\item Técnica estadística: Modelo ESPEGS \cite{Rashid:2008}
	\end{itemize}
\item Capacidad para tomar decisiones razonadas:
	\begin{itemize} 
	\item Autoevaluación y/o evaluación entre iguales \cite{Achcaoucaou:2012}
	\end{itemize}
\item Capacidad para planificar y administrar el tiempo
	\begin{itemize} 
	\item Autoevaluación y/o evaluación entre iguales \cite{Achcaoucaou:2012}, \cite{Liao:2013}
	\end{itemize}
\item Capacidad para adaptarse y actuar en nuevas situaciones
	\begin{itemize} 
	\item Autoevaluación \cite{Liao:2013}
	\end{itemize}
\item La apreciación y el respeto por la diversidad y multiculturalidad:
	\begin{itemize} 
	\item Autoevaluación \cite{Liao:2013}
	\end{itemize}
\item Capacidad de aplicar los conocimientos en situaciones prácticas:
	\begin{itemize} 
	\item Autoevaluación \cite{Liao:2013}
	\end{itemize}
\item Capacidad para identificar, plantear y resolver problemas:
	\begin{itemize} 
	\item Autoevaluación y/o evaluación entre iguales \cite{Achcaoucaou:2012}
	\item Indicadores basados en rúbrica \cite{Guenaga:2013}
	\end{itemize}
\item Habilidad para trabajar de forma autónoma:
	\begin{itemize} 
	\item Autoevaluación y/o evaluación entre iguales \cite{Colomo-Palacios:2013}
	\end{itemize} 
\item Habilidad para trabajar de forma autónoma:
	\begin{itemize} 
	\item Autoevaluación y/o evaluación entre iguales \cite{Colomo-Palacios:2013}
	\end{itemize} 
\item Capacidad para el pensamiento abstracto, análisis y síntesis:
	\begin{itemize} 
	\item Autoevaluación y/o evaluación entre iguales \cite{Colomo-Palacios:2013}
	\end{itemize} 
\item Capacidad para ser críticos y autocríticos:
	\begin{itemize} 
	\item Autoevaluación y/o evaluación entre iguales \cite{Colomo-Palacios:2013}
	\end{itemize} 
\item Espíritu de empresa, la capacidad de tomar la iniciativa:
	\begin{itemize} 
	\item Autoevaluación y/o evaluación entre iguales \cite{Colomo-Palacios:2013}
	\end{itemize} 
\end{itemize}

%-----------
% PREGUNTA 4
%-----------
\subsection{¿Los resultados que se obtendrían de la evaluación de competencias genéricas de los alumnos mediante el uso de los registros de actividad de los entornos virtuales son indicadores fiables del desempeño de dichas competencias?}

No se han encontrado trabajos que aborden esta problemática de manera directa, aunque sí se han encontrado evidencias en la literatura que puedan ayudar a responder a esta pregunta. En alguno de los trabajos se utilizan wikis para fomentar el desempeño de los alumnos en diferentes competencias \cite{Piedra:2010}. Lim indica que los tutores pueden recoger una gran cantidad de información acerca de sus estudiantes mediante la observación de desempeño de los alumnos, mediante la construcción de los wikis, los comentarios realizados por los estudiantes, y los intercambios entre estudiante \cite{Lim:2011}. Pero es evidente que si el curso tiene un gran número de alumnos la observación se vuelve inabordable. Nuestro trabajo va en la línea de observar el trabajo de los alumnos en los LMS mediante la extracción automática de indicadores. Aunque un LMS es más completo en cuánto al ámbito de actuación que un wiki, en ambos los usuarios trabajan en el sistema de forma independiente, dejando un rastro de su actividad y en ambos se sustenta el trabajo colaborativo. Es más, los LMS suelen incluir un wiki en su estructura. \emph{Un wiki es un tipo de página web que brinda la posibilidad de que multitud de usuarios puedan editar sus contenidos a través del navegador web, con ciertas restricciones mínimas. De esta forma permite que múltiples autores puedan crear, modificar o eliminar los contenidos. Se puede identificar a cada usuario que realiza un cambio y recuperar los contenidos modificados, volviendo a un estado anterior. Estas características facilitan el trabajo en colaboración así como la coordinación de acciones e intercambio de información sin necesidad de estar presentes físicamente ni conectados de forma simultánea} (Wikipedia). 

Por último, Cardona dice que la evaluación de las competencias implica la identificación de los elementos en torno a los procesos de aprendizaje, lo que hace que sea una actividad constante que requiere de criterios para evaluar los resultados durante la formación de las personas \cite{Cardona:2013}. Consideramos que el rastro que dejan los alumnos, su interacción con el entorno, es un reflejo de su trabajo. Chebil indica como se podrían analizar las situaciones que se dan en la aplicación de las tecnologías al aprendizaje (TEL, Technology Enhanced Learning) mediante la recopilación de los rastros de interacción producidos por estos entornos \cite{Chebil:2012}. Mientras que Florian argumenta que para poder utilizar esta información almacenada acerca de las actividades del curso, se requiere filtrado antes de que pueda ser utilizado para procesarla \cite{Florian:2011}.


