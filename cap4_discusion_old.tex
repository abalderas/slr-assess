% RESPUESTAS A PREGUNTAS QUITADAS DEL DOCUMENTO

%-----------
% PREGUNTA 1
%-----------
\section{¿Qué competencias se han evaluado de forma automática o asistida por ordenador a partir del uso de los entornos virtuales?}
Las competencias que se han evaluado en los artículos de este trabajo son las siguientes:
\begin{itemize}
\item Capacidad para comunicarse en un segundo idioma \cite{Shih:2011}, \cite{MercedesRico:2013}, \cite{Masip-Alvarez:2013}
\item Capacidad para comunicarse de forma oral y por medio de la palabra escrita en la lengua materna \cite{Mohamed:2008}, \cite{Piedra:2010}, \cite{Liao:2013}, \cite{Masip-Alvarez:2013}, \cite{Colomo-Palacios:2013}, \cite{Bedek:2011}
\item Capacidad para generar nuevas ideas (creatividad) \cite{Piedra:2010}, \cite{Liao:2013}, \cite{Colomo-Palacios:2013}
\item Capacidad para trabajar en equipo \cite{McMahon:2007}, \cite{Mohamed:2008}, \cite{Rashid:2008}, \cite{Lim:2011}, \cite{Liao:2013}, \cite{Masip-Alvarez:2013}, \cite{Colomo-Palacios:2013}, \cite{Gil:2011}, \cite{Palomares:2011}, \cite{Velasco:2012}	
\item Capacidad para tomar decisiones razonadas \cite{Achcaoucaou:2012}, \cite{Colomo-Palacios:2013}, \cite{Borrajo:2010}
\item Capacidad para planificar y administrar el tiempo \cite{Achcaoucaou:2012}, \cite{Liao:2013}
\item Capacidad para adaptarse y actuar en nuevas situaciones \cite{Liao:2013}
\item La apreciación y el respeto por la diversidad y multiculturalidad \cite{Liao:2013}, \cite{Colomo-Palacios:2013}
\item Capacidad de aplicar los conocimientos en situaciones prácticas \cite{Liao:2013}
\item Capacidad para identificar, plantear y resolver problemas \cite{Achcaoucaou:2012}, \cite{Guenaga:2013}, \cite{Colomo-Palacios:2013}
\item Habilidad para trabajar de forma autónoma \cite{Colomo-Palacios:2013}, \cite{Palomares:2011}
\item Capacidad para el pensamiento abstracto, análisis y síntesis \cite{Colomo-Palacios:2013}
\item Capacidad para ser críticos y autocríticos \cite{Colomo-Palacios:2013}
\item Espíritu de empresa, la capacidad de tomar la iniciativa \cite{Colomo-Palacios:2013}
\end{itemize}

%-----------
% PREGUNTA 2
%-----------
\section{¿Qué herramientas o metodologías se utilizan para evaluar competencias mediante el uso de los entornos virtuales?}
Las herramientas o metodologías que se han empleado en nuestra selección primaria para cada competencia son las siguientes:
\begin{itemize}
\item Capacidad para comunicarse en un segundo idioma:
	\begin{itemize} 
	\item Facebook \cite{Shih:2011}
	\item Second Life \cite{MercedesRico:2013}
	\item Grabación en vídeo \cite{Masip-Alvarez:2013}
	\end{itemize}
\item Capacidad para comunicarse de forma oral y por medio de la palabra escrita en la lengua materna:
	\begin{itemize} 
	\item Grabación en vídeo \cite{Masip-Alvarez:2013}
	\item GBL \cite{Bedek:2011}
	\end{itemize}
\item Capacidad para trabajar en equipo:
	\begin{itemize} 
	\item Entorno de seguimiento de trabajo en equipo. JAMTART (Joe And Marks Tracking And Reporting Tool) \cite{McMahon:2007}
	\item Grabación en vídeo \cite{Masip-Alvarez:2013}, \cite{Martin-Cuadrado:2013}
	\item Wikis, Portfolio \cite{Piedra:2010}, \cite{Lim:2011}, \cite{Gil:2011}, \cite{Palomares:2011}, \cite{Velasco:2012}	
	\item DSWC \cite{Liao:2013}
	\end{itemize}
\item Capacidad para tomar decisiones razonadas:
	\begin{itemize} 
	\item Tricuspoid \cite{Achcaoucaou:2012}, se trata de una plataforma digital diseñada específicamente para la auto-evaluación de competencias empresariales.
	\item GBL \cite{Borrajo:2010}
	\end{itemize}
\item Capacidad para planificar y administrar el tiempo
	\begin{itemize} 
	\item Tricuspoid \cite{Achcaoucaou:2012}.
	\end{itemize}
\item Capacidad para generar nuevas ideas (creatividad):
	\begin{itemize} 
	\item Wikis, blogs \cite{Piedra:2010}
	\end{itemize}
\item Capacidad para identificar, plantear y resolver problemas:
	\begin{itemize} 
	\item Tricuspoid \cite{Achcaoucaou:2012}.
	\item Juego serio \cite{Guenaga:2013}
	\item Rasch measurement \cite{Mohamed:2008a} 
	\end{itemize}
\item Habilidad para trabajar de forma autónoma:
	\begin{itemize} 
	\item Portfolio \cite{Palomares:2011}
	\end{itemize} 
\end{itemize}

%-----------
% PREGUNTA 3
%-----------
\section{¿Con qué técnicas se pueden obtener evidencias objetivas del desarrollo de las siguientes competencias en un entorno virtual?}
\begin{itemize}
\item Capacidad para comunicarse en un segundo idioma:
	\begin{itemize} 
	\item Evaluación entre iguales y/o autoevaluación \cite{Shih:2011}, \cite{Masip-Alvarez:2013}
	\item Evaluación conjunta de varios profesores \cite{MercedesRico:2013}
	\end{itemize}
\item Capacidad para comunicarse de forma oral y por medio de la palabra escrita en la lengua materna
	\begin{itemize} 
	\item Autoevaluación \cite{Liao:2013}
	\item Evaluación entre iguales y/o autoevaluación \cite{Masip-Alvarez:2013}
	\item Rúbricas \cite{Mohamed:2008}
	\item eAssessment \cite{Bedek:2011}
	\end{itemize}
\item Capacidad para generar nuevas ideas (creatividad)
	\begin{itemize} 
	\item Autoevaluación \cite{Liao:2013}
	\end{itemize}
\item Capacidad para trabajar en equipo
	\begin{itemize} 
	\item Autoevaluación y/o evaluación entre iguales \cite{McMahon:2007}, \cite{Lim:2011}, \cite{Masip-Alvarez:2013}, \cite{Liao:2013}, \cite{Gil:2011}
	\item Rúbricas \cite{Mohamed:2008}, \cite{Piedra:2010}, \cite{Velasco:2012}	
	\item Técnica estadística: Modelo ESPEGS \cite{Rashid:2008}
	\end{itemize}
\item Capacidad para tomar decisiones razonadas:
	\begin{itemize} 
	\item Autoevaluación y/o evaluación entre iguales \cite{Achcaoucaou:2012}
	\item eAssessment \cite{Borrajo:2010}
	\end{itemize}
\item Capacidad para planificar y administrar el tiempo
	\begin{itemize} 
	\item Autoevaluación y/o evaluación entre iguales \cite{Achcaoucaou:2012}, \cite{Liao:2013}
	\end{itemize}
\item Capacidad para adaptarse y actuar en nuevas situaciones
	\begin{itemize} 
	\item Autoevaluación \cite{Liao:2013}
	\end{itemize}
\item La apreciación y el respeto por la diversidad y multiculturalidad:
	\begin{itemize} 
	\item Autoevaluación \cite{Liao:2013}
	\end{itemize}
\item Capacidad de aplicar los conocimientos en situaciones prácticas:
	\begin{itemize} 
	\item Autoevaluación \cite{Liao:2013}
	\end{itemize}
\item Capacidad para identificar, plantear y resolver problemas:
	\begin{itemize} 
	\item Autoevaluación y/o evaluación entre iguales \cite{Achcaoucaou:2012}
	\item Indicadores basados en rúbrica \cite{Guenaga:2013}
	\end{itemize}
\item Habilidad para trabajar de forma autónoma:
	\begin{itemize} 
	\item Autoevaluación y/o evaluación entre iguales \cite{Colomo-Palacios:2013}
	\item Rúbricas \cite{Palomares:2011}
	\end{itemize} 
\item Habilidad para trabajar de forma autónoma:
	\begin{itemize} 
	\item Autoevaluación y/o evaluación entre iguales \cite{Colomo-Palacios:2013}
	\end{itemize} 
\item Capacidad para el pensamiento abstracto, análisis y síntesis:
	\begin{itemize} 
	\item Autoevaluación y/o evaluación entre iguales \cite{Colomo-Palacios:2013}
	\end{itemize} 
\item Capacidad para ser críticos y autocríticos:
	\begin{itemize} 
	\item Autoevaluación y/o evaluación entre iguales \cite{Colomo-Palacios:2013}
	\end{itemize} 
\item Espíritu de empresa, la capacidad de tomar la iniciativa:
	\begin{itemize} 
	\item Autoevaluación y/o evaluación entre iguales \cite{Colomo-Palacios:2013}
	\end{itemize} 
\end{itemize}

