% -*-cap1.tex-*-
% Este fichero es parte de la plantilla LaTeX para
% la realización de Proyectos Final de Carrera, protejido
% bajo los términos de la licencia GFDL.
% Para más información, la licencia completa viene incluida en el
% fichero fdl-1.3.tex

% Copyright (C) 2009 Pablo Recio Quijano 

%\section{Introducción}

El interés de la evaluación de competencias de aprendizaje ha cambiado del conocimiento a las competencias. Proyectos como el Tuning Educational Structures in Europe, apoyado por el Lifelong Learning Program de la Unión Europea, muestran la importancia de utilizar el concepto de competencia como base para los resultados de aprendizaje. Las competencias de aprendizaje son habilidades que un alumno ha de ser capaz de demostrar una vez que termina su formación. Estas competencias de aprendizaje se dividen en dos grupos: específicas y genéricas. Competencias específicas son aquellas relacionadas directamente con la utilización de conceptos, teorías o habilidades propias de un área en concreto, mientras que las competencias genéricas son habilidades, capacidades y conocimientos que cualquier estudiante debería desarrollar independientemente de su área de estudio \cite{Tuning:2003}. Aunque obviamente sigue siendo muy importante el desarrollo del conocimiento específico de cada área de estudio, es un hecho que el tiempo y la atención también deben dedicarse al desarrollo de las competencias genéricas. También es importante reconocer la aplicación de dichas habilidades genéricas fuera del ámbito académico, ya que son cada vez más relevantes para la preparación de estudiantes para su futuro papel en la sociedad, en términos de empleabilidad y ciudadanía.

Desafortunadamente, evaluar ciertas competencias es a menudo una tarea subjetiva, tanto para el profesor como para los alumnos. A menos que una competencia genérica está directamente enlazada a una actividad específica, éstas son complejas de evaluar. Desarrollar un procedimiento detallado para la evaluación en el desempeño de los estudiantes en las competencias genéricas es una actividad compleja y que requiere mucho tiempo por los diferentes aspectos a tener en cuenta. Si los profesores a penas tienen tiempo sufiente durante el curso académico para cumplir su planificación y evaluar todas las tareas, exámenes o trabajos que los alumnos han tenido que realizar para demostrar la adquisición del conocimiento de la asignatura, más complicado será si además tienen que añadir tareas para evaluar las competencias genéricas. Por lo que, aunque un alumno haya superado una asignatura, no se puede garantizar que éste sea capaz de desempeñar las competencias genéricas que la profesión le va a demandar.

El ámbito de nuestro trabajo está relacionado con los entornos de aprendizaje virtual (LMS, Learning Management System) diseñados especialmente para tareas colaborativas y que son ampliamente utilizados hoy en día por los centros educativos a todos los niveles. En los LMS cada archivo subido, cada acceso al sistema o cada comentario escrito en un foro por los estudiantes quedan registrados en el sistema. Algunos estudios muestran que se podrían analizar las situaciones que se dan en la aplicación de las tecnologías al aprendizaje (TEL, Technology Enhanced Learning) mediante la recopilación de los rastros de interacción producidos por estos entornos \cite{Chebil:2012}. Sin embargo, la recopilación de la información almacenada acerca de las actividades del curso requiere filtrado antes de que pueda ser utilizado para el procesamiento de nivel superior \cite{Florian:2011}. Esta información recopilada podría ser utilizada como indicadores del trabajo de los estudiantes. Los indicadores son estados que determinan si se cumple un nivel para considerar la competencia como alcanzada. Se deben diseñar indicadores para que cada estudiante sea capaz de demostrar su competencia con independencia de su nota. 

En este punto lo que es necesario saber para continuar este trabajo es en qué punto se haya ahora mismo la investigación en lo referente al uso de indicadores extraídos de los LMSs para la evaluación de competencias genéricas. Para ello vamos a realizar una revisión de la literatura, empezando por explicar la metodología en el siguiente capítulo, mostrando los resultados obtenidos en el tercero y las conclusiones en el último.




