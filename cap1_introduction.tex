% INTRODUCCION

El interés de la evaluación de competencias de aprendizaje ha cambiado del conocimiento a las competencias. Proyectos como el Tuning Educational Structures in Europe, apoyado por el Lifelong Learning Program de la Unión Europea, muestran la importancia de utilizar el concepto de competencia como base para los resultados de aprendizaje. Las competencias de aprendizaje son habilidades que un alumno ha de ser capaz de demostrar una vez que termina su formación. Estas competencias de aprendizaje se dividen en dos grupos: específicas y genéricas. Competencias específicas son aquellas relacionadas directamente con la utilización de conceptos, teorías o habilidades propias de un área en concreto, mientras que las competencias genéricas son habilidades, capacidades y conocimientos que cualquier estudiante debería desarrollar independientemente de su área de estudio \cite{Tuning:2003}. Aunque obviamente sigue siendo muy importante el desarrollo del conocimiento específico de cada área de estudio, es un hecho que el tiempo y la atención también deben dedicarse al desarrollo de las competencias genéricas. También es importante reconocer la aplicación de dichas habilidades genéricas fuera del ámbito académico, ya que son cada vez más relevantes para la preparación de estudiantes para su futuro papel en la sociedad, en términos de empleabilidad y ciudadanía.

Desafortunadamente, evaluar ciertas competencias es a menudo una tarea subjetiva, tanto para el profesor como para los alumnos. A menos que una competencia genérica está directamente enlazada a una actividad específica, éstas son difíciles de evaluar. Desarrollar un procedimiento detallado para la evaluación en el desempeño de los estudiantes en las competencias genéricas es una actividad compleja y que requiere mucho tiempo por los diferentes aspectos a tener en cuenta. Si los profesores apenas tienen tiempo suficiente durante el curso académico para cumplir su planificación y evaluar todas las tareas, exámenes o trabajos que los alumnos han tenido que realizar para demostrar la adquisición del conocimiento de la asignatura, más complicado será si además tienen que añadir tareas para evaluar las competencias genéricas. Por lo que, aunque un alumno haya superado una asignatura, no se puede garantizar que éste sea capaz de desempeñar las competencias genéricas que la profesión le va a demandar.

Dado que vivimos en un contexto donde las habilidades demandadas son cambiantes por la rápida transformación en una sociedad basada en el conocimiento, los sistemas de educación y formación en Europa, se es cada vez más conscientes de que los planes de estudio y con ellos las estrategias de evaluación deben centrarse en un enfoque más integral de las competencias genéricas y específicas. Las TIC ofrecen muchas oportunidades para el apoyo a los formatos de evaluación que pueden capturar habilidades complejas y competencias que son difíciles de evaluar \cite{Redecker:2013}. Si los planes de estudio y los objetivos de aprendizaje han cambiado, también deberían hacerlo las prácticas de evaluación \cite{Cachia:2011}.

El ámbito de nuestro trabajo está relacionado con los entornos de aprendizaje virtual: LMS y VLE (LMS, Learning Management System y VLE, Virtual Learning Environment). Estos entornos están diseñados especialmente para tareas colaborativas ya que integran foros, wikis o forjas, y son ampliamente utilizados como un soporte para clases presenciales, haciendo más fácil la comunicación con los estudiantes y manteniendo siempre disponibles las actividades y recursos para el tema \cite{Zafra:2011, Munkhchimeg:2013}. Además, son ampliamente utilizados hoy en día por los centros educativos a todos los niveles.

Cuando el número de alumnos es elevado, se hace mucho menos escalable la evaluación para el profesor. Un ejemplo de este tipo de cursos son los MOOC (Massive Open Online Course), cuya filosofía es la liberación del conocimiento, para que este llegue a un público más amplio, y para el que se suelen ofrecer plazas ilimitadas. Este tipo de curso, que también se caracteriza por ser de carácter abierto y gratuito, y con materiales accesibles de forma gratuita, presenta evidentes problemas de escalabilidad cuando la evaluación no está automatizada \cite{Johnson:2013}.

En los LMS, cada archivo subido, cada actividad realizada, cada acceso al sistema o cada comentario escrito en un foro por los estudiantes quedan registrados en el sistema. ¿Podríamos aprovechar la información almacenada sobre la interacción de los estudiantes con el sistema como indicadores del desempeño de ciertas competencias? Algunos estudios indican que así es, ya que se podrían analizar las situaciones que se dan en la aplicación de las tecnologías al aprendizaje (TEL, Technology Enhanced Learning) mediante la recopilación de los rastros de interacción producidos por estos entornos \cite{Chebil:2012}. Aunque se especifica que esta recopilación de la información almacenada acerca de las actividades del curso requiere filtrado antes de que pueda ser utilizado para el procesamiento de nivel superior \cite{Florian:2011}. 

El objetivo de este trabajo de investigación es realizar un análisis de la literatura para conocer hasta qué punto la ciencia se ha ocupado de la evaluación de competencias genéricas, prestando especial atención en aquellos trabajos que realicen dicha evaluación de manera automatizada aprovechando las prestaciones de las nuevas tecnologías. Nosotros pensamos que la información almacenada en la base de datos de un LMS puede ser aprovechada para extraer indicadores que nos den una medida del desempeño de los alumnos en ciertas competencias genéricas. ¿Hay algo de esto en la literatura? ¿Qué competencias pueden ser evaluadas de manera automática? Todas estas dudas, serán planteadas de manera formal en en el siguiente capítulo, dónde se darán las indicaciones de la metodología seguida. A continuación se mostrarán los resultados, las respuestas a las preguntas, el trabajo futuro y por útlimo las conclusiones y la bibliografía utilizada.




